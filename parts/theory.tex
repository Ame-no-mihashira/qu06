\documentclass[../main]{subfiles}

\begin{document}
\textbf{Ley de la rapidez}\\
Téngase la reacción siguiente:
\begin{equation*}
  \ce{aA + bB -> cC + dD}
\end{equation*}
Su rapidez es descriptible como resultado de la constante de rapidez y las concentraciones de los reactivos:
\begin{equation*}
  v = k \ce{[A]}^{x} \ce{[B]}^{y}
\end{equation*}
La influencia de cada concentración es descrita por cada exponente.
Esta observación es llamada \textit{ley de rapidez}. \parencite{book:chang2013}
\end{document}
